
% Pasur is a traditional card game and requires a standard 52-card deck (no jokers) and is played by 2 to 4 players. In this paper, we focus on the two-player version, with players referred to as \textbf{Alex} and \textbf{Bob} along with the \textbf{Dealer}.

% The game begins with four cards placed face-up on the table to form the initial pool. The initial pool must not contain any Jacks. If a Jack appears among the initial four cards, it is returned to the deck and replaced. If multiple Jacks appear or a replacement is also a Jack, the dealer reshuffles and redeals.

% Once the pool cards are valid and face-up, the dealer deals four cards to each player, starting from the player on their left (assumed to be Alex). The players then take turns, starting with Alex. On each turn, a player must play one card from their hand. The played card will either:

% \begin{itemize}
%     \item Be added to the pool of face-up cards, or
%     \item Capture one or more cards from the pool.
% \end{itemize}

% If capturing is possible, the player \emph{must} capture; they may not simply add a card to the pool. Captured cards are used to calculate the players' scores at the end of the game. Table~\ref{tab:score} shows the scoring system used in Pasur. Notably, a \textbf{Sur}—which is defined below—awards 5 bonus points.

% A \textbf{Sur} occurs when a player captures all the cards from the pool in a single move. However, there are two exceptions:
% \begin{itemize}
%     \item A Sur is not possible using a Jack.
%     \item Surs cannot be scored during the final round of play.
% \end{itemize}

% \begin{table}[h!]
% \centering
% \begin{tabular}{|c|c|}
% \hline
% \textbf{Rule} & \textbf{Points} \\
% \hline
% Most Clubs & 7 \\
% Each Jack & 1 \\
% Each Ace & 1 \\
% Each \;Sur & 5 \\
% 10{\color{red}\ding{117}}& 3 \\
% 2{\ding{168}}& 2 \\
% \hline
% \end{tabular}
% \caption{Pasur Scoring System}
% \label{tab:score}
% \end{table}
% \subsubsection{A Gamplay Example}


% 

% \documentclass[a4paper, 12pt]{article}
% \usepackage[margin=1in]{geometry}
% \usepackage{array}
% \usepackage{longtable}
% \usepackage{booktabs}
% \usepackage{graphicx} % For \resizebox
% % H (red)
\newcommand{\ThH}{\textcolor{red}{3\heartsuit}}  
\newcommand{\TwH}{\textcolor{red}{2\heartsuit}}  
\newcommand{\FoH}{\textcolor{red}{4\heartsuit}}   
\newcommand{\FiH}{\textcolor{red}{5\heartsuit}}   
\newcommand{\SiH}{\textcolor{red}{6\heartsuit}}    
\newcommand{\SeH}{\textcolor{red}{7\heartsuit}}  
\newcommand{\EH}{\textcolor{red}{8\heartsuit}}  
\newcommand{\NH}{\textcolor{red}{9\heartsuit}}   
\newcommand{\TeH}{\textcolor{red}{10\heartsuit}}   
\newcommand{\JH}{\textcolor{red}{J\heartsuit}}    
\newcommand{\QH}{\textcolor{red}{Q\heartsuit}}   
\newcommand{\KH}{\textcolor{red}{K\heartsuit}}    
\newcommand{\AH}{\textcolor{red}{A\heartsuit}}     

% D (red)
\newcommand{\ThD}{\textcolor{red}{3\diamondsuit}}  
\newcommand{\TwD}{\textcolor{red}{2\diamondsuit}}  
\newcommand{\FoD}{\textcolor{red}{4\diamondsuit}}   
\newcommand{\FiD}{\textcolor{red}{5\diamondsuit}}   
\newcommand{\SiD}{\textcolor{red}{6\diamondsuit}}    
\newcommand{\SeD}{\textcolor{red}{7\diamondsuit}}  
\newcommand{\ED}{\textcolor{red}{8\diamondsuit}}  
\newcommand{\ND}{\textcolor{red}{9\diamondsuit}}   
\newcommand{\TeD}{\textcolor{red}{10\diamondsuit}}   
\newcommand{\JD}{\textcolor{red}{J\diamondsuit}}    
\newcommand{\QD}{\textcolor{red}{Q\diamondsuit}}   
\newcommand{\KD}{\textcolor{red}{K\diamondsuit}}    
\newcommand{\AD}{\textcolor{red}{A\diamondsuit}}     

% C (black)
\newcommand{\ThC}{\textcolor{black}{3\clubsuit}}   
\newcommand{\TwC}{\textcolor{black}{2\clubsuit}}   
\newcommand{\FoC}{\textcolor{black}{4\clubsuit}}    
\newcommand{\FiC}{\textcolor{black}{5\clubsuit}}    
\newcommand{\SiC}{\textcolor{black}{6\clubsuit}}     
\newcommand{\SeC}{\textcolor{black}{7\clubsuit}}   
\newcommand{\EC}{\textcolor{black}{8\clubsuit}}   
\newcommand{\NC}{\textcolor{black}{9\clubsuit}}    
\newcommand{\TeC}{\textcolor{black}{10\clubsuit}}    
\newcommand{\JC}{\textcolor{black}{J\clubsuit}}    
\newcommand{\QC}{\textcolor{black}{Q\clubsuit}}   
\newcommand{\KC}{\textcolor{black}{K\clubsuit}}    
\newcommand{\AC}{\textcolor{black}{A\clubsuit}}     

% S (black)
\newcommand{\ThS}{\textcolor{black}{3\spadesuit}}   
\newcommand{\TwS}{\textcolor{black}{2\spadesuit}}   
\newcommand{\FoS}{\textcolor{black}{4\spadesuit}}    
\newcommand{\FiS}{\textcolor{black}{5\spadesuit}}    
\newcommand{\SiS}{\textcolor{black}{6\spadesuit}}     
\newcommand{\SeS}{\textcolor{black}{7\spadesuit}}   
\newcommand{\ES}{\textcolor{black}{8\spadesuit}}   
\newcommand{\NS}{\textcolor{black}{9\spadesuit}}    
\newcommand{\TeS}{\textcolor{black}{10\spadesuit}}    
\newcommand{\JS}{\textcolor{black}{J\spadesuit}}    
\newcommand{\QS}{\textcolor{black}{Q\spadesuit}}   
\newcommand{\KS}{\textcolor{black}{K\spadesuit}}    
\newcommand{\AS}{\textcolor{black}{A\spadesuit}}     

% \usepackage{xcolor}
% \usepackage{amssymb}
% \usepackage{caption}
% \usepackage{colortbl}
% \usepackage{float} % for [H] placement, optional
% \definecolor{myrowcolor}{HTML}{CCFFCC}   
% \begin{document}
% \begin{figure}[H]  % or [htbp] if you want it to float
% \begin{center}
% \resizebox{1\textwidth}{!}{%
% \begin{tabular}{|>{\centering\arraybackslash}p{0.1\textwidth}|
%                 >{\centering\arraybackslash}p{0.4\textwidth}|
%                 >{\centering\arraybackslash}p{0.4\textwidth}|
%                 >{\centering\arraybackslash}p{0.4\textwidth}|
%                 >{\centering\arraybackslash}p{0.3\textwidth}|
%                  >{\centering\arraybackslash}p{0.3\textwidth}|
%                 >{\centering\arraybackslash}p{0.05\textwidth}|
%                 >{\centering\arraybackslash}p{0.05\textwidth}|
%                  >{\centering\arraybackslash}p{0.05\textwidth}|
%                   >{\centering\arraybackslash}p{0.05\textwidth}|
%                    >{\centering\arraybackslash}p{0.05\textwidth}|
%                    >{\centering\arraybackslash}p{0.05\textwidth}|
%                    >{\centering\arraybackslash}p{0.05\textwidth}|
%                    >{\centering\arraybackslash}p{0.05\textwidth}|}
% \hline
% \textbf{Stage}& \textbf{Alex} & \textbf{Bob}& \textbf{Pool} & \textbf{Lay} & \textbf{Pick} & \textbf{Acl} &  \textbf{Bcl} & \textbf{Apt} &  \textbf{Bpt} & \textbf{Asr} & \textbf{Bsr} & \textbf{$\Delta$} & \textbf{L}\\
% \hline

%  0\_0\_0 &$\FiS\;\SiC\;\NS\;\QD$&$\TwD\;\TwS\;\SiD\;\ED$&$\ThD\;\SiH\;\SeC\;\QS$&$\FiS$&$\SiH$&0 &0 &0 &0 &0 &0 &0 &A\\ 
 
%  \hline


%  0\_0\_1 &$\SiC\;\NS\;\QD$&$\TwD\;\TwS\;\SiD\;\ED$&$\ThD\;\SeC\;\QS$&$\ED$&$\ThD$&0 &0 &0 &0 &0 &0 &0 &B\\ \hline

%  0\_1\_0 &$\SiC\;\NS\;\QD$&$\TwD\;\TwS\;\SiD$&$\SeC\;\QS$&$\NS$&&0 &0 &0 &0 &0 &0 &0 &B\\ \hline

%  0\_1\_1 &$\SiC\;\QD$&$\TwD\;\TwS\;\SiD$&$\SeC\;\NS\;\QS$&$\TwD$&$\NS$&0 &0 &0 &0 &0 &0 &0 &B\\ \hline

%  0\_2\_0 &$\SiC\;\QD$&$\TwS\;\SiD$&$\SeC\;\QS$&$\QD$&$\QS$&0 &0 &0 &0 &0 &0 &0 &A\\ \hline

%  0\_2\_1 &$\SiC$&$\TwS\;\SiD$&$\SeC$&$\TwS$&&0 &0 &0 &0 &0 &0 &0 &A\\ \hline

%  0\_3\_0 &$\SiC$&$\SiD$&$\TwS\;\SeC$&$\SiC$&&0 &0 &0 &0 &0 &0 &0 &A\\ \hline

%  0\_3\_1 &&$\SiD$&$\TwS\;\SiC\;\SeC$&$\SiD$&&0 &0 &0 &0 &0 &0 &0 &A\\ \hline
% \hline 


%  1\_0\_0 &$\SeD\;\TeC\;\TeS\;\JD$&$\AC\;\ThH\;\SeH\;\NC$&$\TwS\;\SiC\;\SiD\;\SeC$&$\TeC$&&0 &0 &0 &0 &0 &0 &0 &0\\ \hline

%  1\_0\_1 &$\SeD\;\TeS\;\JD$&$\AC\;\ThH\;\SeH\;\NC$&$\TwS\;\SiC\;\SiD\;\SeC\;\TeC$&$\ThH$&$\TwS\;\SiD$&0 &0 &0 &0 &0 &0 &0 &B\\ \hline

%  1\_1\_0 &$\SeD\;\TeS\;\JD$&$\AC\;\SeH\;\NC$&$\SiC\;\SeC\;\TeC$&$\JD$&$\SiC\;\SeC\;\TeC$&3 &0 &1 &0 &0 &0 &0 &A\\ \hline

%  1\_1\_1 &$\SeD\;\TeS$&$\AC\;\SeH\;\NC$&&$\NC$&&3 &0 &1 &0 &0 &0 &0 &A\\ \hline

%  1\_2\_0 &$\SeD\;\TeS$&$\AC\;\SeH$&$\NC$&$\SeD$&&3 &0 &1 &0 &0 &0 &0 &A\\ \hline

%  1\_2\_1 &$\TeS$&$\AC\;\SeH$&$\SeD\;\NC$&$\SeH$&&3 &0 &1 &0 &0 &0 &0 &A\\ \hline

%  1\_3\_0 &$\TeS$&$\AC$&$\SeD\;\SeH\;\NC$&$\TeS$&&3 &0 &1 &0 &0 &0 &0 &A\\ \hline

%  1\_3\_1 &&$\AC$&$\SeD\;\SeH\;\NC\;\TeS$&$\AC$&$\TeS$&3 &1 &1 &1 &0 &0 &0 &B\\ \hline
% \hline 


%  2\_0\_0 &$\FoH\;\FoS\;\QC\;\KS$&$\AH\;\FiH\;\SiS\;\JC$&$\SeD\;\SeH\;\NC$&$\FoS$&$\SeD$&0 &0 &0 &0 &0 &0 &0 &A\\ \hline

%  2\_0\_1 &$\FoH\;\QC\;\KS$&$\AH\;\FiH\;\SiS\;\JC$&$\SeH\;\NC$&$\SiS$&&0 &0 &0 &0 &0 &0 &0 &A\\ \hline

%  2\_1\_0 &$\FoH\;\QC\;\KS$&$\AH\;\FiH\;\JC$&$\SiS\;\SeH\;\NC$&$\FoH$&$\SeH$&0 &0 &0 &0 &0 &0 &0 &A\\ \hline

%  2\_1\_1 &$\QC\;\KS$&$\AH\;\FiH\;\JC$&$\SiS\;\NC$&$\AH$&&0 &0 &0 &0 &0 &0 &0 &A\\ \hline

%  2\_2\_0 &$\QC\;\KS$&$\FiH\;\JC$&$\AH\;\SiS\;\NC$&$\QC$&&0 &0 &0 &0 &0 &0 &0 &A\\ \hline

%  2\_2\_1 &$\KS$&$\FiH\;\JC$&$\AH\;\SiS\;\NC\;\QC$&$\FiH$&$\SiS$&0 &0 &0 &0 &0 &0 &0 &B\\ \hline

%  2\_3\_0 &$\KS$&$\JC$&$\AH\;\NC\;\QC$&$\KS$&&0 &0 &0 &0 &0 &0 &0 &B\\ \hline

%  2\_3\_1 &&$\JC$&$\AH\;\NC\;\QC\;\KS$&$\JC$&$\AH\;\NC$&0 &2 &0 &2 &0 &0 &0 &B\\ \hline
% \hline 


%  3\_0\_0 &$\AD\;\AS\;\FoC\;\NH$&$\FoD\;\TeD\;\QH\;\KD$&$\QC\;\KS$&$\AS$&&0 &0 &0 &0 &0 &0 &0 &0\\ \hline

%  3\_0\_1 &$\AD\;\FoC\;\NH$&$\FoD\;\TeD\;\QH\;\KD$&$\AS\;\QC\;\KS$&$\TeD$&$\AS$&0 &0 &0 &4 &0 &0 &0 &B\\ \hline

%  3\_1\_0 &$\AD\;\FoC\;\NH$&$\FoD\;\QH\;\KD$&$\QC\;\KS$&$\FoC$&&0 &0 &0 &4 &0 &0 &0 &B\\ \hline

%  3\_1\_1 &$\AD\;\NH$&$\FoD\;\QH\;\KD$&$\FoC\;\QC\;\KS$&$\FoD$&&0 &0 &0 &4 &0 &0 &0 &B\\ \hline

%  3\_2\_0 &$\AD\;\NH$&$\QH\;\KD$&$\FoC\;\FoD\;\QC\;\KS$&$\NH$&&0 &0 &0 &4 &0 &0 &0 &B\\ \hline

%  3\_2\_1 &$\AD$&$\QH\;\KD$&$\FoC\;\FoD\;\NH\;\QC\;\KS$&$\QH$&$\QC$&0 &1 &0 &4 &0 &0 &0 &B\\ \hline

%  3\_3\_0 &$\AD$&$\KD$&$\FoC\;\FoD\;\NH\;\KS$&$\AD$&&0 &1 &0 &4 &0 &0 &0 &B\\ \hline

%  3\_3\_1 &&$\KD$&$\AD\;\FoC\;\FoD\;\NH\;\KS$&$\KD$&$\KS$&0 &1 &0 &4 &0 &0 &0 &B\\ \hline
% \hline 


%  4\_0\_0 &$\TwC\;\SeS\;\EH\;\KH$&$\ThC\;\ThS\;\FiD\;\ND$&$\AD\;\FoC\;\FoD\;\NH$&$\SeS$&$\FoC$&1 &0 &0 &0 &0 &0 &0 &A\\ \hline

%  4\_0\_1 &$\TwC\;\EH\;\KH$&$\ThC\;\ThS\;\FiD\;\ND$&$\AD\;\FoD\;\NH$&$\ThS$&&1 &0 &0 &0 &0 &0 &0 &A\\ \hline

%  4\_1\_0 &$\TwC\;\EH\;\KH$&$\ThC\;\FiD\;\ND$&$\AD\;\ThS\;\FoD\;\NH$&$\KH$&&1 &0 &0 &0 &0 &0 &0 &A\\ \hline

%  4\_1\_1 &$\TwC\;\EH$&$\ThC\;\FiD\;\ND$&$\AD\;\ThS\;\FoD\;\NH\;\KH$&$\ThC$&$\AD\;\ThS\;\FoD$&1 &1 &0 &1 &0 &0 &0 &B\\ \hline

%  4\_2\_0 &$\TwC\;\EH$&$\FiD\;\ND$&$\NH\;\KH$&$\TwC$&$\NH$&2 &1 &2 &1 &0 &0 &0 &A\\ \hline

%  4\_2\_1 &$\EH$&$\FiD\;\ND$&$\KH$&$\FiD$&&2 &1 &2 &1 &0 &0 &0 &A\\ \hline

%  4\_3\_0 &$\EH$&$\ND$&$\FiD\;\KH$&$\EH$&&2 &1 &2 &1 &0 &0 &0 &A\\ \hline

%  4\_3\_1 &&$\ND$&$\FiD\;\EH\;\KH$&$\ND$&&2 &1 &2 &1 &0 &0 &0 &A\\ \hline
% \hline 


%  5\_0\_0 &$\FiC\;\EC\;\ES\;\JS$&$\TwH\;\TeH\;\JH\;\KC$&$\FiD\;\EH\;\ND\;\KH$&$\FiC$&&0 &0 &0 &0 &0 &0 &0 &0\\ \hline

%  5\_0\_1 &$\EC\;\ES\;\JS$&$\TwH\;\TeH\;\JH\;\KC$&$\FiC\;\FiD\;\EH\;\ND\;\KH$&$\TwH$&$\ND$&0 &0 &0 &0 &0 &0 &0 &B\\ \hline

%  5\_1\_0 &$\EC\;\ES\;\JS$&$\TeH\;\JH\;\KC$&$\FiC\;\FiD\;\EH\;\KH$&$\ES$&&0 &0 &0 &0 &0 &0 &0 &B\\ \hline

%  5\_1\_1 &$\EC\;\JS$&$\TeH\;\JH\;\KC$&$\FiC\;\FiD\;\EH\;\ES\;\KH$&$\JH$&$\FiC\;\FiD\;\EH\;\ES$&0 &1 &0 &1 &0 &0 &0 &B\\ \hline

%  5\_2\_0 &$\EC\;\JS$&$\TeH\;\KC$&$\KH$&$\EC$&&0 &1 &0 &1 &0 &0 &0 &B\\ \hline

%  5\_2\_1 &$\JS$&$\TeH\;\KC$&$\EC\;\KH$&$\KC$&$\KH$&0 &2 &0 &1 &0 &0 &0 &B\\ \hline

%  5\_3\_0 &$\JS$&$\TeH$&$\EC$&$\JS$&$\EC$&1 &2 &1 &1 &0 &0 &0 &A\\ \hline

%  5\_3\_1 &&$\TeH$&&$\TeH$&&1 &2 &1 &1 &0 &0 &0 &A\\ \hline
% \hline 

% CleanUp &&&&&&1 &2 &1 &1 &0 &0 &0 &A\\ \hline
% \end{tabular}
% }
% \end{center}
%     \caption{This is a table wrapped inside a figure environment.}
%     \label{fig:my_table}
% \end{figure}

% \end{document}


\begin{figure}[!p]  % or [htbp] if you want it to float
\caption*{Pasur: A Step-by-Step Gameplay Snapshot}
\begin{center}
\resizebox{1\textwidth}{!}{%
\begin{tabular}{|>{\centering\arraybackslash}p{0.1\textwidth}|
                >{\centering\arraybackslash}p{0.2\textwidth}|
                >{\centering\arraybackslash}p{0.2\textwidth}|
                >{\centering\arraybackslash}p{0.7\textwidth}|
                >{\centering\arraybackslash}p{0.05\textwidth}|
                 >{\centering\arraybackslash}p{0.7\textwidth}|
                >{\centering\arraybackslash}p{0.05\textwidth}|
                >{\centering\arraybackslash}p{0.05\textwidth}|
                 >{\centering\arraybackslash}p{0.05\textwidth}|
                  >{\centering\arraybackslash}p{0.05\textwidth}|
                   >{\centering\arraybackslash}p{0.05\textwidth}|
                   >{\centering\arraybackslash}p{0.05\textwidth}|
                   >{\centering\arraybackslash}p{0.05\textwidth}|
                   >{\centering\arraybackslash}p{0.05\textwidth}|>{\centering\arraybackslash}p{0.05\textwidth}|}
\hline
\textbf{Stage}& \textbf{Alex} & \textbf{Bob}& \textbf{Pool} & \textbf{Lay} & \textbf{Pick} & \textbf{Acl} &  \textbf{Bcl} & \textbf{Apt} &  \textbf{Bpt} & \textbf{Asr} & \textbf{Bsr} & \textbf{$\Delta$} & \textbf{L} & \textbf{CL}\\
\hline
\rowcolor{myrowcolor}


 0\_0\_0 &$\FoC\;\FoD\;\SeD\;\QC$&$\ThD\;\ThH\;\FiC\;\KS$&$\AC\;\AS\;\ND\;\KD$&$\FoD$&&0 &0 &0 &0 &0 &0 &0 &0&-\\ \hline

 0\_0\_1 &$\FoC\;\SeD\;\QC$&$\ThD\;\ThH\;\FiC\;\KS$&$\AC\;\AS\;\FoD\;\ND\;\KD$&$\KS$&$\KD$&0 &0 &0 &0 &0 &0 &0 &B&-\\ \hline
 \rowcolor{myrowcolor}

 0\_1\_0 &$\FoC\;\SeD\;\QC$&$\ThD\;\ThH\;\FiC$&$\AC\;\AS\;\FoD\;\ND$&$\QC$&&0 &0 &0 &0 &0 &0 &0 &B&-\\ \hline

 0\_1\_1 &$\FoC\;\SeD$&$\ThD\;\ThH\;\FiC$&$\AC\;\AS\;\FoD\;\ND\;\QC$&$\FiC$&$\AC\;\AS\;\FoD$&0 &2 &0 &2 &0 &0 &0 &B&-\\ \hline
 \rowcolor{myrowcolor}

 0\_2\_0 &$\FoC\;\SeD$&$\ThD\;\ThH$&$\ND\;\QC$&$\FoC$&&0 &2 &0 &2 &0 &0 &0 &B&-\\ \hline

 0\_2\_1 &$\SeD$&$\ThD\;\ThH$&$\FoC\;\ND\;\QC$&$\ThD$&&0 &2 &0 &2 &0 &0 &0 &B&-\\ \hline
 \rowcolor{myrowcolor}

 0\_3\_0 &$\SeD$&$\ThH$&$\ThD\;\FoC\;\ND\;\QC$&$\SeD$&$\FoC$&1 &2 &0 &2 &0 &0 &0 &A&-\\ \hline

 0\_3\_1 &&$\ThH$&$\ThD\;\ND\;\QC$&$\ThH$&&1 &2 &0 &2 &0 &0 &0 &A&-\\ \hline
 \rowcolor{myrowcolor}
\hline 


 1\_0\_0 &$\SiC\;\SiD\;\NH\;\JC$&$\SiH\;\SeC\;\JD\;\KH$&$\ThD\;\ThH\;\ND\;\QC$&$\JC$&$\ThD\;\ThH\;\ND$&2 &2 &1 &0 &0 &0 &-2 &A&-\\ \hline

 1\_0\_1 &$\SiC\;\SiD\;\NH$&$\SiH\;\SeC\;\JD\;\KH$&$\QC$&$\JD$&&2 &2 &1 &0 &0 &0 &-2 &A&-\\ \hline
 \rowcolor{myrowcolor}

 1\_1\_0 &$\SiC\;\SiD\;\NH$&$\SiH\;\SeC\;\KH$&$\JD\;\QC$&$\SiC$&&2 &2 &1 &0 &0 &0 &-2 &A&-\\ \hline

 1\_1\_1 &$\SiD\;\NH$&$\SiH\;\SeC\;\KH$&$\SiC\;\JD\;\QC$&$\KH$&&2 &2 &1 &0 &0 &0 &-2 &A&-\\ \hline
 \rowcolor{myrowcolor}

 1\_2\_0 &$\SiD\;\NH$&$\SiH\;\SeC$&$\SiC\;\JD\;\QC\;\KH$&$\SiD$&&2 &2 &1 &0 &0 &0 &-2 &A&-\\ \hline

 1\_2\_1 &$\NH$&$\SiH\;\SeC$&$\SiC\;\SiD\;\JD\;\QC\;\KH$&$\SeC$&&2 &2 &1 &0 &0 &0 &-2 &A&-\\ \hline
 \rowcolor{myrowcolor}

 1\_3\_0 &$\NH$&$\SiH$&$\SiC\;\SiD\;\SeC\;\JD\;\QC\;\KH$&$\NH$&&2 &2 &1 &0 &0 &0 &-2 &A&-\\ \hline

 1\_3\_1 &&$\SiH$&$\SiC\;\SiD\;\SeC\;\NH\;\JD\;\QC\;\KH$&$\SiH$&&2 &2 &1 &0 &0 &0 &-2 &A&-\\ \hline
 \rowcolor{myrowcolor}
\hline 


 2\_0\_0 &$\FiD\;\NC\;\TeH\;\KC$&$\AD\;\SiS\;\EC\;\ES$&$\SiC\;\SiD\;\SiH\;\SeC\;\NH\;\JD\;\QC\;\KH$&$\TeH$&&2 &2 &0 &0 &0 &0 &-1 &0&-\\ \hline

 2\_0\_1 &$\FiD\;\NC\;\KC$&$\AD\;\SiS\;\EC\;\ES$&$\SiC\;\SiD\;\SiH\;\SeC\;\NH\;\TeH\;\JD\;\QC\;\KH$&$\EC$&&2 &2 &0 &0 &0 &0 &-1 &0&-\\ \hline
 \rowcolor{myrowcolor}

 2\_1\_0 &$\FiD\;\NC\;\KC$&$\AD\;\SiS\;\ES$&$\SiC\;\SiD\;\SiH\;\SeC\;\EC\;\NH\;\TeH\;\JD\;\QC\;\KH$&$\KC$&$\KH$&3 &2 &0 &0 &0 &0 &-1 &A&-\\ \hline

 2\_1\_1 &$\FiD\;\NC$&$\AD\;\SiS\;\ES$&$\SiC\;\SiD\;\SiH\;\SeC\;\EC\;\NH\;\TeH\;\JD\;\QC$&$\AD$&$\TeH$&3 &2 &0 &1 &0 &0 &-1 &B&-\\ \hline
 \rowcolor{myrowcolor}

 2\_2\_0 &$\FiD\;\NC$&$\SiS\;\ES$&$\SiC\;\SiD\;\SiH\;\SeC\;\EC\;\NH\;\JD\;\QC$&$\FiD$&$\SiH$&3 &2 &0 &1 &0 &0 &-1 &A&-\\ \hline

 2\_2\_1 &$\NC$&$\SiS\;\ES$&$\SiC\;\SiD\;\SeC\;\EC\;\NH\;\JD\;\QC$&$\ES$&&3 &2 &0 &1 &0 &0 &-1 &A&-\\ \hline
 \rowcolor{myrowcolor}

 2\_3\_0 &$\NC$&$\SiS$&$\SiC\;\SiD\;\SeC\;\EC\;\ES\;\NH\;\JD\;\QC$&$\NC$&&3 &2 &0 &1 &0 &0 &-1 &A&-\\ \hline

 2\_3\_1 &&$\SiS$&$\SiC\;\SiD\;\SeC\;\EC\;\ES\;\NC\;\NH\;\JD\;\QC$&$\SiS$&&3 &2 &0 &1 &0 &0 &-1 &A&-\\ \hline
 \rowcolor{myrowcolor}
\hline 


 3\_0\_0 &$\TwC\;\ThC\;\JS\;\QS$&$\SeS\;\ED\;\EH\;\QH$&$\SiC\;\SiD\;\SiS\;\SeC\;\EC\;\ES\;\NC\;\NH\;\JD\;\QC$&$\TwC$&$\NC$&5 &2 &2 &0 &0 &0 &-2 &A&-\\ \hline

 3\_0\_1 &$\ThC\;\JS\;\QS$&$\SeS\;\ED\;\EH\;\QH$&$\SiC\;\SiD\;\SiS\;\SeC\;\EC\;\ES\;\NH\;\JD\;\QC$&$\ED$&&5 &2 &2 &0 &0 &0 &-2 &A&-\\ \hline
 \rowcolor{myrowcolor}

 3\_1\_0 &$\ThC\;\JS\;\QS$&$\SeS\;\EH\;\QH$&$\SiC\;\SiD\;\SiS\;\SeC\;\EC\;\ED\;\ES\;\NH\;\JD\;\QC$&$\JS$&$\SiC\;\SiD\;\SiS\;\SeC\;\EC\;\ED\;\ES\;\NH\;\JD$&8 &2 &4 &0 &0 &0 &-2 &A&-\\ \hline

 3\_1\_1 &$\ThC\;\QS$&$\SeS\;\EH\;\QH$&$\QC$&$\SeS$&&8 &2 &4 &0 &0 &0 &-2 &A&-\\ \hline
 \rowcolor{myrowcolor}

 3\_2\_0 &$\ThC\;\QS$&$\EH\;\QH$&$\SeS\;\QC$&$\ThC$&&8 &2 &4 &0 &0 &0 &-2 &A&-\\ \hline

 3\_2\_1 &$\QS$&$\EH\;\QH$&$\ThC\;\SeS\;\QC$&$\QH$&$\QC$&8 &3 &4 &0 &0 &0 &-2 &B&-\\ \hline
 \rowcolor{myrowcolor}

 3\_3\_0 &$\QS$&$\EH$&$\ThC\;\SeS$&$\QS$&&8 &3 &4 &0 &0 &0 &-2 &B&-\\ \hline

 3\_3\_1 &&$\EH$&$\ThC\;\SeS\;\QS$&$\EH$&$\ThC$&8 &4 &4 &0 &0 &0 &-2 &B&-\\ \hline
 \rowcolor{myrowcolor}
\hline 


 4\_0\_0 &$\FoH\;\FoS\;\SeH\;\JH$&$\TwD\;\FiH\;\NS\;\TeS$&$\SeS\;\QS$&$\FoH$&$\SeS$&0 &0 &0 &0 &0 &0 &2 &A&A\\ \hline

 4\_0\_1 &$\FoS\;\SeH\;\JH$&$\TwD\;\FiH\;\NS\;\TeS$&$\QS$&$\FiH$&&0 &0 &0 &0 &0 &0 &2 &A&A\\ \hline
 \rowcolor{myrowcolor}

 4\_1\_0 &$\FoS\;\SeH\;\JH$&$\TwD\;\NS\;\TeS$&$\FiH\;\QS$&$\SeH$&&0 &0 &0 &0 &0 &0 &2 &A&A\\ \hline

 4\_1\_1 &$\FoS\;\JH$&$\TwD\;\NS\;\TeS$&$\FiH\;\SeH\;\QS$&$\TeS$&&0 &0 &0 &0 &0 &0 &2 &A&A\\ \hline
 \rowcolor{myrowcolor}

 4\_2\_0 &$\FoS\;\JH$&$\TwD\;\NS$&$\FiH\;\SeH\;\TeS\;\QS$&$\JH$&$\FiH\;\SeH\;\TeS$&0 &0 &1 &0 &0 &0 &2 &A&A\\ \hline

 4\_2\_1 &$\FoS$&$\TwD\;\NS$&$\QS$&$\TwD$&&0 &0 &1 &0 &0 &0 &2 &A&A\\ \hline
 \rowcolor{myrowcolor}

 4\_3\_0 &$\FoS$&$\NS$&$\TwD\;\QS$&$\FoS$&&0 &0 &1 &0 &0 &0 &2 &A&A\\ \hline

 4\_3\_1 &&$\NS$&$\TwD\;\FoS\;\QS$&$\NS$&$\TwD$&0 &0 &1 &0 &0 &0 &2 &B&A\\ \hline
 \rowcolor{myrowcolor}
\hline 


 5\_0\_0 &$\AH\;\ThS\;\TeC\;\QD$&$\TwH\;\TwS\;\FiS\;\TeD$&$\FoS\;\QS$&$\AH$&&0 &0 &0 &0 &0 &0 &3 &0&A\\ \hline

 5\_0\_1 &$\ThS\;\TeC\;\QD$&$\TwH\;\TwS\;\FiS\;\TeD$&$\AH\;\FoS\;\QS$&$\FiS$&&0 &0 &0 &0 &0 &0 &3 &0&A\\ \hline
 \rowcolor{myrowcolor}

 5\_1\_0 &$\ThS\;\TeC\;\QD$&$\TwH\;\TwS\;\TeD$&$\AH\;\FoS\;\FiS\;\QS$&$\TeC$&$\AH$&1 &0 &1 &0 &0 &0 &3 &A&A\\ \hline

 5\_1\_1 &$\ThS\;\QD$&$\TwH\;\TwS\;\TeD$&$\FoS\;\FiS\;\QS$&$\TeD$&&1 &0 &1 &0 &0 &0 &3 &A&A\\ \hline
 \rowcolor{myrowcolor}

 5\_2\_0 &$\ThS\;\QD$&$\TwH\;\TwS$&$\FoS\;\FiS\;\TeD\;\QS$&$\QD$&$\QS$&1 &0 &1 &0 &0 &0 &3 &A&A\\ \hline

 5\_2\_1 &$\ThS$&$\TwH\;\TwS$&$\FoS\;\FiS\;\TeD$&$\TwS$&$\FoS\;\FiS$&1 &0 &1 &0 &0 &0 &3 &B&A\\ \hline
 \rowcolor{myrowcolor}

 5\_3\_0 &$\ThS$&$\TwH$&$\TeD$&$\ThS$&&1 &0 &1 &0 &0 &0 &3 &B&A\\ \hline

 5\_3\_1 &&$\TwH$&$\ThS\;\TeD$&$\TwH$&&1 &0 &1 &0 &0 &0 &3 &B&A\\ \hline
\hline 

CleanUp &&&&&&1 &0 &1 &3 &0 &0 &2 &B&A\\ \hline
\end{tabular}
}
\end{center}
\caption{Columns \texttt{Acl}, \texttt{Apt}, and \texttt{Asr} represent the number of clubs, points, and surs collected or earned by Alex so far. \texttt{Bcl}, \texttt{Bpt}, and \texttt{Bsr} are defined similarly for Bob.  Once, at the end of any round, \texttt{Acl} exceeds 7, both \texttt{Acl} and \texttt{Bcl} reset to zero, and the column \texttt{CL} indicates which player collected at least 7 clubs. According to Pasur rules, this player earns 7 points. Collecting more than 7 clubs yields no additional points unless the card is also a point card (i.e., \textit{e.g.}, $\AC$ or 2\ding{168} ). The column $\Delta$ shows the cumulative point difference up to the end of the previous round. $\Delta$ updates at the end of each round to reflect the points earned in that round. Finally, in the \textit{CleanUp} phase, the player who made the last pick (as shown in column \texttt{L}) collects all remaining cards from the pool. If any point cards are present, the corresponding point columns will be updated. If there are club cards in the CleanUp phase and neither player has yet reached 7 clubs, then the remaining clubs in the pool determine who earns the 7-club bonus. In such situations, the identity of the last player to pick becomes critical. 
\\
\\
In the displayed game above, Alex earns the 7-club bonus. Bob gains 3 points from collecting $\TeD$, but he is trailing by 3 points from the rounds preceding the last. Additionally, Alex earns 1 point in the final round from capturing the $\AH$ card. This results in a final score with Alex leading by 8 points. See Table~\ref{tab:score} for the Pasur scoring system.
}
\label{fig:game_0}
\end{figure}

\newpage


Pasur is a traditional card game played with a standard 52-card deck (excluding jokers), and it supports 2 to 4 players. In this paper, we focus on the two-player variant, where the players are referred to as \textbf{Alex} and \textbf{Bob}, along with a \textbf{Dealer} who manages the game.

The game begins with four cards placed face-up on the table to form the initial pool. This pool must not contain any Jacks. If a Jack appears among the initial four cards, it is returned to the deck and replaced. If multiple Jacks are dealt, or if a replacement card is also a Jack, the dealer reshuffles and redeals.

Once the pool is valid and face-up, the dealer deals four cards to each player, starting with the player on their left (assumed to be Alex). Players then take turns beginning with Alex. On each turn, a player must play one card from their hand. The played card will either:

\begin{itemize}
    \item Be added to the pool of face-up cards, or
    \item Capture one or more cards from the pool, following these rules:
    \begin{itemize}
        \item A numeric card can capture a combination of numeric cards from the pool if their total sum equals 11.
        \item A Queen can only capture another single Queen. The same rule applies to Kings.
        \item A Jack captures all cards in the pool (including other Jacks), except for Kings and Queens.
    \end{itemize}
\end{itemize}

If a capture is possible, the player \emph{must} capture; they cannot simply add a card to the pool. Captured cards are retained and used to calculate each player’s score at the end of the game. Table~\ref{tab:score} outlines the scoring system used in Pasur. Notably, a \textbf{Sur}—defined below—grants a 5-point bonus.

A \textbf{Sur} occurs when a player captures all the cards from the pool in a single move. There are two important exceptions:
\begin{itemize}
    \item A Sur cannot be made using a Jack.
    \item Surs are not permitted during the final round of play.
\end{itemize}

\begin{table}[H]
\centering
\begin{tabular}{|c|c|}
\hline
\textbf{Rule} & \textbf{Points} \\
\hline
Most Clubs & 7 \\
Each Jack & 1 \\
Each Ace & 1 \\
Each \;Sur & 5 \\
10{\color{red}\ding{117}} & 3 \\
2{\ding{168}} & 2 \\
\hline
\end{tabular}
\caption{Pasur Scoring System}
\label{tab:score}
\end{table}



% \documentclass[a4paper, 12pt]{article}
% \usepackage[margin=1in]{geometry}
% \usepackage{array}
% \usepackage{longtable}
% \usepackage{booktabs}
% \usepackage{graphicx} % For \resizebox
% % H (red)
\newcommand{\ThH}{\textcolor{red}{3\heartsuit}}  
\newcommand{\TwH}{\textcolor{red}{2\heartsuit}}  
\newcommand{\FoH}{\textcolor{red}{4\heartsuit}}   
\newcommand{\FiH}{\textcolor{red}{5\heartsuit}}   
\newcommand{\SiH}{\textcolor{red}{6\heartsuit}}    
\newcommand{\SeH}{\textcolor{red}{7\heartsuit}}  
\newcommand{\EH}{\textcolor{red}{8\heartsuit}}  
\newcommand{\NH}{\textcolor{red}{9\heartsuit}}   
\newcommand{\TeH}{\textcolor{red}{10\heartsuit}}   
\newcommand{\JH}{\textcolor{red}{J\heartsuit}}    
\newcommand{\QH}{\textcolor{red}{Q\heartsuit}}   
\newcommand{\KH}{\textcolor{red}{K\heartsuit}}    
\newcommand{\AH}{\textcolor{red}{A\heartsuit}}     

% D (red)
\newcommand{\ThD}{\textcolor{red}{3\diamondsuit}}  
\newcommand{\TwD}{\textcolor{red}{2\diamondsuit}}  
\newcommand{\FoD}{\textcolor{red}{4\diamondsuit}}   
\newcommand{\FiD}{\textcolor{red}{5\diamondsuit}}   
\newcommand{\SiD}{\textcolor{red}{6\diamondsuit}}    
\newcommand{\SeD}{\textcolor{red}{7\diamondsuit}}  
\newcommand{\ED}{\textcolor{red}{8\diamondsuit}}  
\newcommand{\ND}{\textcolor{red}{9\diamondsuit}}   
\newcommand{\TeD}{\textcolor{red}{10\diamondsuit}}   
\newcommand{\JD}{\textcolor{red}{J\diamondsuit}}    
\newcommand{\QD}{\textcolor{red}{Q\diamondsuit}}   
\newcommand{\KD}{\textcolor{red}{K\diamondsuit}}    
\newcommand{\AD}{\textcolor{red}{A\diamondsuit}}     

% C (black)
\newcommand{\ThC}{\textcolor{black}{3\clubsuit}}   
\newcommand{\TwC}{\textcolor{black}{2\clubsuit}}   
\newcommand{\FoC}{\textcolor{black}{4\clubsuit}}    
\newcommand{\FiC}{\textcolor{black}{5\clubsuit}}    
\newcommand{\SiC}{\textcolor{black}{6\clubsuit}}     
\newcommand{\SeC}{\textcolor{black}{7\clubsuit}}   
\newcommand{\EC}{\textcolor{black}{8\clubsuit}}   
\newcommand{\NC}{\textcolor{black}{9\clubsuit}}    
\newcommand{\TeC}{\textcolor{black}{10\clubsuit}}    
\newcommand{\JC}{\textcolor{black}{J\clubsuit}}    
\newcommand{\QC}{\textcolor{black}{Q\clubsuit}}   
\newcommand{\KC}{\textcolor{black}{K\clubsuit}}    
\newcommand{\AC}{\textcolor{black}{A\clubsuit}}     

% S (black)
\newcommand{\ThS}{\textcolor{black}{3\spadesuit}}   
\newcommand{\TwS}{\textcolor{black}{2\spadesuit}}   
\newcommand{\FoS}{\textcolor{black}{4\spadesuit}}    
\newcommand{\FiS}{\textcolor{black}{5\spadesuit}}    
\newcommand{\SiS}{\textcolor{black}{6\spadesuit}}     
\newcommand{\SeS}{\textcolor{black}{7\spadesuit}}   
\newcommand{\ES}{\textcolor{black}{8\spadesuit}}   
\newcommand{\NS}{\textcolor{black}{9\spadesuit}}    
\newcommand{\TeS}{\textcolor{black}{10\spadesuit}}    
\newcommand{\JS}{\textcolor{black}{J\spadesuit}}    
\newcommand{\QS}{\textcolor{black}{Q\spadesuit}}   
\newcommand{\KS}{\textcolor{black}{K\spadesuit}}    
\newcommand{\AS}{\textcolor{black}{A\spadesuit}}     

% \usepackage{xcolor}
% \usepackage{amssymb}
% \usepackage{caption}
% \usepackage{colortbl}
% \usepackage{float} % for [H] placement, optional
% \definecolor{myrowcolor}{HTML}{CCFFCC}   
% \begin{document}
% \begin{figure}[H]  % or [htbp] if you want it to float
% \begin{center}
% \resizebox{1\textwidth}{!}{%
% \begin{tabular}{|>{\centering\arraybackslash}p{0.1\textwidth}|
%                 >{\centering\arraybackslash}p{0.4\textwidth}|
%                 >{\centering\arraybackslash}p{0.4\textwidth}|
%                 >{\centering\arraybackslash}p{0.4\textwidth}|
%                 >{\centering\arraybackslash}p{0.3\textwidth}|
%                  >{\centering\arraybackslash}p{0.3\textwidth}|
%                 >{\centering\arraybackslash}p{0.05\textwidth}|
%                 >{\centering\arraybackslash}p{0.05\textwidth}|
%                  >{\centering\arraybackslash}p{0.05\textwidth}|
%                   >{\centering\arraybackslash}p{0.05\textwidth}|
%                    >{\centering\arraybackslash}p{0.05\textwidth}|
%                    >{\centering\arraybackslash}p{0.05\textwidth}|
%                    >{\centering\arraybackslash}p{0.05\textwidth}|
%                    >{\centering\arraybackslash}p{0.05\textwidth}|}
% \hline
% \textbf{Stage}& \textbf{Alex} & \textbf{Bob}& \textbf{Pool} & \textbf{Lay} & \textbf{Pick} & \textbf{Acl} &  \textbf{Bcl} & \textbf{Apt} &  \textbf{Bpt} & \textbf{Asr} & \textbf{Bsr} & \textbf{$\Delta$} & \textbf{L}\\
% \hline

%  0\_0\_0 &$\FiS\;\SiC\;\NS\;\QD$&$\TwD\;\TwS\;\SiD\;\ED$&$\ThD\;\SiH\;\SeC\;\QS$&$\FiS$&$\SiH$&0 &0 &0 &0 &0 &0 &0 &A\\ 
 
%  \hline


%  0\_0\_1 &$\SiC\;\NS\;\QD$&$\TwD\;\TwS\;\SiD\;\ED$&$\ThD\;\SeC\;\QS$&$\ED$&$\ThD$&0 &0 &0 &0 &0 &0 &0 &B\\ \hline

%  0\_1\_0 &$\SiC\;\NS\;\QD$&$\TwD\;\TwS\;\SiD$&$\SeC\;\QS$&$\NS$&&0 &0 &0 &0 &0 &0 &0 &B\\ \hline

%  0\_1\_1 &$\SiC\;\QD$&$\TwD\;\TwS\;\SiD$&$\SeC\;\NS\;\QS$&$\TwD$&$\NS$&0 &0 &0 &0 &0 &0 &0 &B\\ \hline

%  0\_2\_0 &$\SiC\;\QD$&$\TwS\;\SiD$&$\SeC\;\QS$&$\QD$&$\QS$&0 &0 &0 &0 &0 &0 &0 &A\\ \hline

%  0\_2\_1 &$\SiC$&$\TwS\;\SiD$&$\SeC$&$\TwS$&&0 &0 &0 &0 &0 &0 &0 &A\\ \hline

%  0\_3\_0 &$\SiC$&$\SiD$&$\TwS\;\SeC$&$\SiC$&&0 &0 &0 &0 &0 &0 &0 &A\\ \hline

%  0\_3\_1 &&$\SiD$&$\TwS\;\SiC\;\SeC$&$\SiD$&&0 &0 &0 &0 &0 &0 &0 &A\\ \hline
% \hline 


%  1\_0\_0 &$\SeD\;\TeC\;\TeS\;\JD$&$\AC\;\ThH\;\SeH\;\NC$&$\TwS\;\SiC\;\SiD\;\SeC$&$\TeC$&&0 &0 &0 &0 &0 &0 &0 &0\\ \hline

%  1\_0\_1 &$\SeD\;\TeS\;\JD$&$\AC\;\ThH\;\SeH\;\NC$&$\TwS\;\SiC\;\SiD\;\SeC\;\TeC$&$\ThH$&$\TwS\;\SiD$&0 &0 &0 &0 &0 &0 &0 &B\\ \hline

%  1\_1\_0 &$\SeD\;\TeS\;\JD$&$\AC\;\SeH\;\NC$&$\SiC\;\SeC\;\TeC$&$\JD$&$\SiC\;\SeC\;\TeC$&3 &0 &1 &0 &0 &0 &0 &A\\ \hline

%  1\_1\_1 &$\SeD\;\TeS$&$\AC\;\SeH\;\NC$&&$\NC$&&3 &0 &1 &0 &0 &0 &0 &A\\ \hline

%  1\_2\_0 &$\SeD\;\TeS$&$\AC\;\SeH$&$\NC$&$\SeD$&&3 &0 &1 &0 &0 &0 &0 &A\\ \hline

%  1\_2\_1 &$\TeS$&$\AC\;\SeH$&$\SeD\;\NC$&$\SeH$&&3 &0 &1 &0 &0 &0 &0 &A\\ \hline

%  1\_3\_0 &$\TeS$&$\AC$&$\SeD\;\SeH\;\NC$&$\TeS$&&3 &0 &1 &0 &0 &0 &0 &A\\ \hline

%  1\_3\_1 &&$\AC$&$\SeD\;\SeH\;\NC\;\TeS$&$\AC$&$\TeS$&3 &1 &1 &1 &0 &0 &0 &B\\ \hline
% \hline 


%  2\_0\_0 &$\FoH\;\FoS\;\QC\;\KS$&$\AH\;\FiH\;\SiS\;\JC$&$\SeD\;\SeH\;\NC$&$\FoS$&$\SeD$&0 &0 &0 &0 &0 &0 &0 &A\\ \hline

%  2\_0\_1 &$\FoH\;\QC\;\KS$&$\AH\;\FiH\;\SiS\;\JC$&$\SeH\;\NC$&$\SiS$&&0 &0 &0 &0 &0 &0 &0 &A\\ \hline

%  2\_1\_0 &$\FoH\;\QC\;\KS$&$\AH\;\FiH\;\JC$&$\SiS\;\SeH\;\NC$&$\FoH$&$\SeH$&0 &0 &0 &0 &0 &0 &0 &A\\ \hline

%  2\_1\_1 &$\QC\;\KS$&$\AH\;\FiH\;\JC$&$\SiS\;\NC$&$\AH$&&0 &0 &0 &0 &0 &0 &0 &A\\ \hline

%  2\_2\_0 &$\QC\;\KS$&$\FiH\;\JC$&$\AH\;\SiS\;\NC$&$\QC$&&0 &0 &0 &0 &0 &0 &0 &A\\ \hline

%  2\_2\_1 &$\KS$&$\FiH\;\JC$&$\AH\;\SiS\;\NC\;\QC$&$\FiH$&$\SiS$&0 &0 &0 &0 &0 &0 &0 &B\\ \hline

%  2\_3\_0 &$\KS$&$\JC$&$\AH\;\NC\;\QC$&$\KS$&&0 &0 &0 &0 &0 &0 &0 &B\\ \hline

%  2\_3\_1 &&$\JC$&$\AH\;\NC\;\QC\;\KS$&$\JC$&$\AH\;\NC$&0 &2 &0 &2 &0 &0 &0 &B\\ \hline
% \hline 


%  3\_0\_0 &$\AD\;\AS\;\FoC\;\NH$&$\FoD\;\TeD\;\QH\;\KD$&$\QC\;\KS$&$\AS$&&0 &0 &0 &0 &0 &0 &0 &0\\ \hline

%  3\_0\_1 &$\AD\;\FoC\;\NH$&$\FoD\;\TeD\;\QH\;\KD$&$\AS\;\QC\;\KS$&$\TeD$&$\AS$&0 &0 &0 &4 &0 &0 &0 &B\\ \hline

%  3\_1\_0 &$\AD\;\FoC\;\NH$&$\FoD\;\QH\;\KD$&$\QC\;\KS$&$\FoC$&&0 &0 &0 &4 &0 &0 &0 &B\\ \hline

%  3\_1\_1 &$\AD\;\NH$&$\FoD\;\QH\;\KD$&$\FoC\;\QC\;\KS$&$\FoD$&&0 &0 &0 &4 &0 &0 &0 &B\\ \hline

%  3\_2\_0 &$\AD\;\NH$&$\QH\;\KD$&$\FoC\;\FoD\;\QC\;\KS$&$\NH$&&0 &0 &0 &4 &0 &0 &0 &B\\ \hline

%  3\_2\_1 &$\AD$&$\QH\;\KD$&$\FoC\;\FoD\;\NH\;\QC\;\KS$&$\QH$&$\QC$&0 &1 &0 &4 &0 &0 &0 &B\\ \hline

%  3\_3\_0 &$\AD$&$\KD$&$\FoC\;\FoD\;\NH\;\KS$&$\AD$&&0 &1 &0 &4 &0 &0 &0 &B\\ \hline

%  3\_3\_1 &&$\KD$&$\AD\;\FoC\;\FoD\;\NH\;\KS$&$\KD$&$\KS$&0 &1 &0 &4 &0 &0 &0 &B\\ \hline
% \hline 


%  4\_0\_0 &$\TwC\;\SeS\;\EH\;\KH$&$\ThC\;\ThS\;\FiD\;\ND$&$\AD\;\FoC\;\FoD\;\NH$&$\SeS$&$\FoC$&1 &0 &0 &0 &0 &0 &0 &A\\ \hline

%  4\_0\_1 &$\TwC\;\EH\;\KH$&$\ThC\;\ThS\;\FiD\;\ND$&$\AD\;\FoD\;\NH$&$\ThS$&&1 &0 &0 &0 &0 &0 &0 &A\\ \hline

%  4\_1\_0 &$\TwC\;\EH\;\KH$&$\ThC\;\FiD\;\ND$&$\AD\;\ThS\;\FoD\;\NH$&$\KH$&&1 &0 &0 &0 &0 &0 &0 &A\\ \hline

%  4\_1\_1 &$\TwC\;\EH$&$\ThC\;\FiD\;\ND$&$\AD\;\ThS\;\FoD\;\NH\;\KH$&$\ThC$&$\AD\;\ThS\;\FoD$&1 &1 &0 &1 &0 &0 &0 &B\\ \hline

%  4\_2\_0 &$\TwC\;\EH$&$\FiD\;\ND$&$\NH\;\KH$&$\TwC$&$\NH$&2 &1 &2 &1 &0 &0 &0 &A\\ \hline

%  4\_2\_1 &$\EH$&$\FiD\;\ND$&$\KH$&$\FiD$&&2 &1 &2 &1 &0 &0 &0 &A\\ \hline

%  4\_3\_0 &$\EH$&$\ND$&$\FiD\;\KH$&$\EH$&&2 &1 &2 &1 &0 &0 &0 &A\\ \hline

%  4\_3\_1 &&$\ND$&$\FiD\;\EH\;\KH$&$\ND$&&2 &1 &2 &1 &0 &0 &0 &A\\ \hline
% \hline 


%  5\_0\_0 &$\FiC\;\EC\;\ES\;\JS$&$\TwH\;\TeH\;\JH\;\KC$&$\FiD\;\EH\;\ND\;\KH$&$\FiC$&&0 &0 &0 &0 &0 &0 &0 &0\\ \hline

%  5\_0\_1 &$\EC\;\ES\;\JS$&$\TwH\;\TeH\;\JH\;\KC$&$\FiC\;\FiD\;\EH\;\ND\;\KH$&$\TwH$&$\ND$&0 &0 &0 &0 &0 &0 &0 &B\\ \hline

%  5\_1\_0 &$\EC\;\ES\;\JS$&$\TeH\;\JH\;\KC$&$\FiC\;\FiD\;\EH\;\KH$&$\ES$&&0 &0 &0 &0 &0 &0 &0 &B\\ \hline

%  5\_1\_1 &$\EC\;\JS$&$\TeH\;\JH\;\KC$&$\FiC\;\FiD\;\EH\;\ES\;\KH$&$\JH$&$\FiC\;\FiD\;\EH\;\ES$&0 &1 &0 &1 &0 &0 &0 &B\\ \hline

%  5\_2\_0 &$\EC\;\JS$&$\TeH\;\KC$&$\KH$&$\EC$&&0 &1 &0 &1 &0 &0 &0 &B\\ \hline

%  5\_2\_1 &$\JS$&$\TeH\;\KC$&$\EC\;\KH$&$\KC$&$\KH$&0 &2 &0 &1 &0 &0 &0 &B\\ \hline

%  5\_3\_0 &$\JS$&$\TeH$&$\EC$&$\JS$&$\EC$&1 &2 &1 &1 &0 &0 &0 &A\\ \hline

%  5\_3\_1 &&$\TeH$&&$\TeH$&&1 &2 &1 &1 &0 &0 &0 &A\\ \hline
% \hline 

% CleanUp &&&&&&1 &2 &1 &1 &0 &0 &0 &A\\ \hline
% \end{tabular}
% }
% \end{center}
%     \caption{This is a table wrapped inside a figure environment.}
%     \label{fig:my_table}
% \end{figure}

% \end{document}


\begin{figure}[!p]  % or [htbp] if you want it to float
\caption*{Pasur: A Step-by-Step Gameplay Snapshot}
\begin{center}
\resizebox{1\textwidth}{!}{%
\begin{tabular}{|>{\centering\arraybackslash}p{0.1\textwidth}|
                >{\centering\arraybackslash}p{0.2\textwidth}|
                >{\centering\arraybackslash}p{0.2\textwidth}|
                >{\centering\arraybackslash}p{0.7\textwidth}|
                >{\centering\arraybackslash}p{0.05\textwidth}|
                 >{\centering\arraybackslash}p{0.7\textwidth}|
                >{\centering\arraybackslash}p{0.05\textwidth}|
                >{\centering\arraybackslash}p{0.05\textwidth}|
                 >{\centering\arraybackslash}p{0.05\textwidth}|
                  >{\centering\arraybackslash}p{0.05\textwidth}|
                   >{\centering\arraybackslash}p{0.05\textwidth}|
                   >{\centering\arraybackslash}p{0.05\textwidth}|
                   >{\centering\arraybackslash}p{0.05\textwidth}|
                   >{\centering\arraybackslash}p{0.05\textwidth}|>{\centering\arraybackslash}p{0.05\textwidth}|}
\hline
\textbf{Stage}& \textbf{Alex} & \textbf{Bob}& \textbf{Pool} & \textbf{Lay} & \textbf{Pick} & \textbf{Acl} &  \textbf{Bcl} & \textbf{Apt} &  \textbf{Bpt} & \textbf{Asr} & \textbf{Bsr} & \textbf{$\Delta$} & \textbf{L} & \textbf{CL}\\
\hline
\rowcolor{myrowcolor}


 0\_0\_0 &$\FoC\;\FoD\;\SeD\;\QC$&$\ThD\;\ThH\;\FiC\;\KS$&$\AC\;\AS\;\ND\;\KD$&$\FoD$&&0 &0 &0 &0 &0 &0 &0 &0&-\\ \hline

 0\_0\_1 &$\FoC\;\SeD\;\QC$&$\ThD\;\ThH\;\FiC\;\KS$&$\AC\;\AS\;\FoD\;\ND\;\KD$&$\KS$&$\KD$&0 &0 &0 &0 &0 &0 &0 &B&-\\ \hline
 \rowcolor{myrowcolor}

 0\_1\_0 &$\FoC\;\SeD\;\QC$&$\ThD\;\ThH\;\FiC$&$\AC\;\AS\;\FoD\;\ND$&$\QC$&&0 &0 &0 &0 &0 &0 &0 &B&-\\ \hline

 0\_1\_1 &$\FoC\;\SeD$&$\ThD\;\ThH\;\FiC$&$\AC\;\AS\;\FoD\;\ND\;\QC$&$\FiC$&$\AC\;\AS\;\FoD$&0 &2 &0 &2 &0 &0 &0 &B&-\\ \hline
 \rowcolor{myrowcolor}

 0\_2\_0 &$\FoC\;\SeD$&$\ThD\;\ThH$&$\ND\;\QC$&$\FoC$&&0 &2 &0 &2 &0 &0 &0 &B&-\\ \hline

 0\_2\_1 &$\SeD$&$\ThD\;\ThH$&$\FoC\;\ND\;\QC$&$\ThD$&&0 &2 &0 &2 &0 &0 &0 &B&-\\ \hline
 \rowcolor{myrowcolor}

 0\_3\_0 &$\SeD$&$\ThH$&$\ThD\;\FoC\;\ND\;\QC$&$\SeD$&$\FoC$&1 &2 &0 &2 &0 &0 &0 &A&-\\ \hline

 0\_3\_1 &&$\ThH$&$\ThD\;\ND\;\QC$&$\ThH$&&1 &2 &0 &2 &0 &0 &0 &A&-\\ \hline
 \rowcolor{myrowcolor}
\hline 


 1\_0\_0 &$\SiC\;\SiD\;\NH\;\JC$&$\SiH\;\SeC\;\JD\;\KH$&$\ThD\;\ThH\;\ND\;\QC$&$\JC$&$\ThD\;\ThH\;\ND$&2 &2 &1 &0 &0 &0 &-2 &A&-\\ \hline

 1\_0\_1 &$\SiC\;\SiD\;\NH$&$\SiH\;\SeC\;\JD\;\KH$&$\QC$&$\JD$&&2 &2 &1 &0 &0 &0 &-2 &A&-\\ \hline
 \rowcolor{myrowcolor}

 1\_1\_0 &$\SiC\;\SiD\;\NH$&$\SiH\;\SeC\;\KH$&$\JD\;\QC$&$\SiC$&&2 &2 &1 &0 &0 &0 &-2 &A&-\\ \hline

 1\_1\_1 &$\SiD\;\NH$&$\SiH\;\SeC\;\KH$&$\SiC\;\JD\;\QC$&$\KH$&&2 &2 &1 &0 &0 &0 &-2 &A&-\\ \hline
 \rowcolor{myrowcolor}

 1\_2\_0 &$\SiD\;\NH$&$\SiH\;\SeC$&$\SiC\;\JD\;\QC\;\KH$&$\SiD$&&2 &2 &1 &0 &0 &0 &-2 &A&-\\ \hline

 1\_2\_1 &$\NH$&$\SiH\;\SeC$&$\SiC\;\SiD\;\JD\;\QC\;\KH$&$\SeC$&&2 &2 &1 &0 &0 &0 &-2 &A&-\\ \hline
 \rowcolor{myrowcolor}

 1\_3\_0 &$\NH$&$\SiH$&$\SiC\;\SiD\;\SeC\;\JD\;\QC\;\KH$&$\NH$&&2 &2 &1 &0 &0 &0 &-2 &A&-\\ \hline

 1\_3\_1 &&$\SiH$&$\SiC\;\SiD\;\SeC\;\NH\;\JD\;\QC\;\KH$&$\SiH$&&2 &2 &1 &0 &0 &0 &-2 &A&-\\ \hline
 \rowcolor{myrowcolor}
\hline 


 2\_0\_0 &$\FiD\;\NC\;\TeH\;\KC$&$\AD\;\SiS\;\EC\;\ES$&$\SiC\;\SiD\;\SiH\;\SeC\;\NH\;\JD\;\QC\;\KH$&$\TeH$&&2 &2 &0 &0 &0 &0 &-1 &0&-\\ \hline

 2\_0\_1 &$\FiD\;\NC\;\KC$&$\AD\;\SiS\;\EC\;\ES$&$\SiC\;\SiD\;\SiH\;\SeC\;\NH\;\TeH\;\JD\;\QC\;\KH$&$\EC$&&2 &2 &0 &0 &0 &0 &-1 &0&-\\ \hline
 \rowcolor{myrowcolor}

 2\_1\_0 &$\FiD\;\NC\;\KC$&$\AD\;\SiS\;\ES$&$\SiC\;\SiD\;\SiH\;\SeC\;\EC\;\NH\;\TeH\;\JD\;\QC\;\KH$&$\KC$&$\KH$&3 &2 &0 &0 &0 &0 &-1 &A&-\\ \hline

 2\_1\_1 &$\FiD\;\NC$&$\AD\;\SiS\;\ES$&$\SiC\;\SiD\;\SiH\;\SeC\;\EC\;\NH\;\TeH\;\JD\;\QC$&$\AD$&$\TeH$&3 &2 &0 &1 &0 &0 &-1 &B&-\\ \hline
 \rowcolor{myrowcolor}

 2\_2\_0 &$\FiD\;\NC$&$\SiS\;\ES$&$\SiC\;\SiD\;\SiH\;\SeC\;\EC\;\NH\;\JD\;\QC$&$\FiD$&$\SiH$&3 &2 &0 &1 &0 &0 &-1 &A&-\\ \hline

 2\_2\_1 &$\NC$&$\SiS\;\ES$&$\SiC\;\SiD\;\SeC\;\EC\;\NH\;\JD\;\QC$&$\ES$&&3 &2 &0 &1 &0 &0 &-1 &A&-\\ \hline
 \rowcolor{myrowcolor}

 2\_3\_0 &$\NC$&$\SiS$&$\SiC\;\SiD\;\SeC\;\EC\;\ES\;\NH\;\JD\;\QC$&$\NC$&&3 &2 &0 &1 &0 &0 &-1 &A&-\\ \hline

 2\_3\_1 &&$\SiS$&$\SiC\;\SiD\;\SeC\;\EC\;\ES\;\NC\;\NH\;\JD\;\QC$&$\SiS$&&3 &2 &0 &1 &0 &0 &-1 &A&-\\ \hline
 \rowcolor{myrowcolor}
\hline 


 3\_0\_0 &$\TwC\;\ThC\;\JS\;\QS$&$\SeS\;\ED\;\EH\;\QH$&$\SiC\;\SiD\;\SiS\;\SeC\;\EC\;\ES\;\NC\;\NH\;\JD\;\QC$&$\TwC$&$\NC$&5 &2 &2 &0 &0 &0 &-2 &A&-\\ \hline

 3\_0\_1 &$\ThC\;\JS\;\QS$&$\SeS\;\ED\;\EH\;\QH$&$\SiC\;\SiD\;\SiS\;\SeC\;\EC\;\ES\;\NH\;\JD\;\QC$&$\ED$&&5 &2 &2 &0 &0 &0 &-2 &A&-\\ \hline
 \rowcolor{myrowcolor}

 3\_1\_0 &$\ThC\;\JS\;\QS$&$\SeS\;\EH\;\QH$&$\SiC\;\SiD\;\SiS\;\SeC\;\EC\;\ED\;\ES\;\NH\;\JD\;\QC$&$\JS$&$\SiC\;\SiD\;\SiS\;\SeC\;\EC\;\ED\;\ES\;\NH\;\JD$&8 &2 &4 &0 &0 &0 &-2 &A&-\\ \hline

 3\_1\_1 &$\ThC\;\QS$&$\SeS\;\EH\;\QH$&$\QC$&$\SeS$&&8 &2 &4 &0 &0 &0 &-2 &A&-\\ \hline
 \rowcolor{myrowcolor}

 3\_2\_0 &$\ThC\;\QS$&$\EH\;\QH$&$\SeS\;\QC$&$\ThC$&&8 &2 &4 &0 &0 &0 &-2 &A&-\\ \hline

 3\_2\_1 &$\QS$&$\EH\;\QH$&$\ThC\;\SeS\;\QC$&$\QH$&$\QC$&8 &3 &4 &0 &0 &0 &-2 &B&-\\ \hline
 \rowcolor{myrowcolor}

 3\_3\_0 &$\QS$&$\EH$&$\ThC\;\SeS$&$\QS$&&8 &3 &4 &0 &0 &0 &-2 &B&-\\ \hline

 3\_3\_1 &&$\EH$&$\ThC\;\SeS\;\QS$&$\EH$&$\ThC$&8 &4 &4 &0 &0 &0 &-2 &B&-\\ \hline
 \rowcolor{myrowcolor}
\hline 


 4\_0\_0 &$\FoH\;\FoS\;\SeH\;\JH$&$\TwD\;\FiH\;\NS\;\TeS$&$\SeS\;\QS$&$\FoH$&$\SeS$&0 &0 &0 &0 &0 &0 &2 &A&A\\ \hline

 4\_0\_1 &$\FoS\;\SeH\;\JH$&$\TwD\;\FiH\;\NS\;\TeS$&$\QS$&$\FiH$&&0 &0 &0 &0 &0 &0 &2 &A&A\\ \hline
 \rowcolor{myrowcolor}

 4\_1\_0 &$\FoS\;\SeH\;\JH$&$\TwD\;\NS\;\TeS$&$\FiH\;\QS$&$\SeH$&&0 &0 &0 &0 &0 &0 &2 &A&A\\ \hline

 4\_1\_1 &$\FoS\;\JH$&$\TwD\;\NS\;\TeS$&$\FiH\;\SeH\;\QS$&$\TeS$&&0 &0 &0 &0 &0 &0 &2 &A&A\\ \hline
 \rowcolor{myrowcolor}

 4\_2\_0 &$\FoS\;\JH$&$\TwD\;\NS$&$\FiH\;\SeH\;\TeS\;\QS$&$\JH$&$\FiH\;\SeH\;\TeS$&0 &0 &1 &0 &0 &0 &2 &A&A\\ \hline

 4\_2\_1 &$\FoS$&$\TwD\;\NS$&$\QS$&$\TwD$&&0 &0 &1 &0 &0 &0 &2 &A&A\\ \hline
 \rowcolor{myrowcolor}

 4\_3\_0 &$\FoS$&$\NS$&$\TwD\;\QS$&$\FoS$&&0 &0 &1 &0 &0 &0 &2 &A&A\\ \hline

 4\_3\_1 &&$\NS$&$\TwD\;\FoS\;\QS$&$\NS$&$\TwD$&0 &0 &1 &0 &0 &0 &2 &B&A\\ \hline
 \rowcolor{myrowcolor}
\hline 


 5\_0\_0 &$\AH\;\ThS\;\TeC\;\QD$&$\TwH\;\TwS\;\FiS\;\TeD$&$\FoS\;\QS$&$\AH$&&0 &0 &0 &0 &0 &0 &3 &0&A\\ \hline

 5\_0\_1 &$\ThS\;\TeC\;\QD$&$\TwH\;\TwS\;\FiS\;\TeD$&$\AH\;\FoS\;\QS$&$\FiS$&&0 &0 &0 &0 &0 &0 &3 &0&A\\ \hline
 \rowcolor{myrowcolor}

 5\_1\_0 &$\ThS\;\TeC\;\QD$&$\TwH\;\TwS\;\TeD$&$\AH\;\FoS\;\FiS\;\QS$&$\TeC$&$\AH$&1 &0 &1 &0 &0 &0 &3 &A&A\\ \hline

 5\_1\_1 &$\ThS\;\QD$&$\TwH\;\TwS\;\TeD$&$\FoS\;\FiS\;\QS$&$\TeD$&&1 &0 &1 &0 &0 &0 &3 &A&A\\ \hline
 \rowcolor{myrowcolor}

 5\_2\_0 &$\ThS\;\QD$&$\TwH\;\TwS$&$\FoS\;\FiS\;\TeD\;\QS$&$\QD$&$\QS$&1 &0 &1 &0 &0 &0 &3 &A&A\\ \hline

 5\_2\_1 &$\ThS$&$\TwH\;\TwS$&$\FoS\;\FiS\;\TeD$&$\TwS$&$\FoS\;\FiS$&1 &0 &1 &0 &0 &0 &3 &B&A\\ \hline
 \rowcolor{myrowcolor}

 5\_3\_0 &$\ThS$&$\TwH$&$\TeD$&$\ThS$&&1 &0 &1 &0 &0 &0 &3 &B&A\\ \hline

 5\_3\_1 &&$\TwH$&$\ThS\;\TeD$&$\TwH$&&1 &0 &1 &0 &0 &0 &3 &B&A\\ \hline
\hline 

CleanUp &&&&&&1 &0 &1 &3 &0 &0 &2 &B&A\\ \hline
\end{tabular}
}
\end{center}
\caption{Columns \texttt{Acl}, \texttt{Apt}, and \texttt{Asr} represent the number of clubs, points, and surs collected or earned by Alex so far. \texttt{Bcl}, \texttt{Bpt}, and \texttt{Bsr} are defined similarly for Bob.  Once, at the end of any round, \texttt{Acl} exceeds 7, both \texttt{Acl} and \texttt{Bcl} reset to zero, and the column \texttt{CL} indicates which player collected at least 7 clubs. According to Pasur rules, this player earns 7 points. Collecting more than 7 clubs yields no additional points unless the card is also a point card (i.e., \textit{e.g.}, $\AC$ or 2\ding{168} ). The column $\Delta$ shows the cumulative point difference up to the end of the previous round. $\Delta$ updates at the end of each round to reflect the points earned in that round. Finally, in the \textit{CleanUp} phase, the player who made the last pick (as shown in column \texttt{L}) collects all remaining cards from the pool. If any point cards are present, the corresponding point columns will be updated. If there are club cards in the CleanUp phase and neither player has yet reached 7 clubs, then the remaining clubs in the pool determine who earns the 7-club bonus. In such situations, the identity of the last player to pick becomes critical. 
\\
\\
In the displayed game above, Alex earns the 7-club bonus. Bob gains 3 points from collecting $\TeD$, but he is trailing by 3 points from the rounds preceding the last. Additionally, Alex earns 1 point in the final round from capturing the $\AH$ card. This results in a final score with Alex leading by 8 points. See Table~\ref{tab:score} for the Pasur scoring system.
}
\label{fig:game_0}
\end{figure}

\newpage


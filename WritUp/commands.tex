%%%%%%% START OF USEPACKAGE %%%%%%%%

\usepackage{algorithm}
\usepackage{algpseudocode}
\usepackage{xcolor}
\usepackage{amsmath}
\usepackage{tikz}
\usepackage{pifont}
\usepackage{bm}
\usepackage{hyperref}
\usepackage{booktabs}
\usepackage{wrapfig}
\usepackage{algorithm}
\usepackage{algpseudocode}
\usepackage{caption}
\usepackage{amssymb}  % For card symbols
% \documentclass[a4paper, 12pt]{article}
% \usepackage[margin=1in]{geometry}
% \usepackage{array}
\usepackage{longtable}
\usepackage{booktabs}
\usepackage{graphicx} % For \resizebox
% H (red)
\newcommand{\ThH}{\textcolor{red}{3\heartsuit}}  
\newcommand{\TwH}{\textcolor{red}{2\heartsuit}}  
\newcommand{\FoH}{\textcolor{red}{4\heartsuit}}   
\newcommand{\FiH}{\textcolor{red}{5\heartsuit}}   
\newcommand{\SiH}{\textcolor{red}{6\heartsuit}}    
\newcommand{\SeH}{\textcolor{red}{7\heartsuit}}  
\newcommand{\EH}{\textcolor{red}{8\heartsuit}}  
\newcommand{\NH}{\textcolor{red}{9\heartsuit}}   
\newcommand{\TeH}{\textcolor{red}{10\heartsuit}}   
\newcommand{\JH}{\textcolor{red}{J\heartsuit}}    
\newcommand{\QH}{\textcolor{red}{Q\heartsuit}}   
\newcommand{\KH}{\textcolor{red}{K\heartsuit}}    
\newcommand{\AH}{\textcolor{red}{A\heartsuit}}     

% D (red)
\newcommand{\ThD}{\textcolor{red}{3\diamondsuit}}  
\newcommand{\TwD}{\textcolor{red}{2\diamondsuit}}  
\newcommand{\FoD}{\textcolor{red}{4\diamondsuit}}   
\newcommand{\FiD}{\textcolor{red}{5\diamondsuit}}   
\newcommand{\SiD}{\textcolor{red}{6\diamondsuit}}    
\newcommand{\SeD}{\textcolor{red}{7\diamondsuit}}  
\newcommand{\ED}{\textcolor{red}{8\diamondsuit}}  
\newcommand{\ND}{\textcolor{red}{9\diamondsuit}}   
\newcommand{\TeD}{\textcolor{red}{10\diamondsuit}}   
\newcommand{\JD}{\textcolor{red}{J\diamondsuit}}    
\newcommand{\QD}{\textcolor{red}{Q\diamondsuit}}   
\newcommand{\KD}{\textcolor{red}{K\diamondsuit}}    
\newcommand{\AD}{\textcolor{red}{A\diamondsuit}}     

% C (black)
\newcommand{\ThC}{\textcolor{black}{3\clubsuit}}   
\newcommand{\TwC}{\textcolor{black}{2\clubsuit}}   
\newcommand{\FoC}{\textcolor{black}{4\clubsuit}}    
\newcommand{\FiC}{\textcolor{black}{5\clubsuit}}    
\newcommand{\SiC}{\textcolor{black}{6\clubsuit}}     
\newcommand{\SeC}{\textcolor{black}{7\clubsuit}}   
\newcommand{\EC}{\textcolor{black}{8\clubsuit}}   
\newcommand{\NC}{\textcolor{black}{9\clubsuit}}    
\newcommand{\TeC}{\textcolor{black}{10\clubsuit}}    
\newcommand{\JC}{\textcolor{black}{J\clubsuit}}    
\newcommand{\QC}{\textcolor{black}{Q\clubsuit}}   
\newcommand{\KC}{\textcolor{black}{K\clubsuit}}    
\newcommand{\AC}{\textcolor{black}{A\clubsuit}}     

% S (black)
\newcommand{\ThS}{\textcolor{black}{3\spadesuit}}   
\newcommand{\TwS}{\textcolor{black}{2\spadesuit}}   
\newcommand{\FoS}{\textcolor{black}{4\spadesuit}}    
\newcommand{\FiS}{\textcolor{black}{5\spadesuit}}    
\newcommand{\SiS}{\textcolor{black}{6\spadesuit}}     
\newcommand{\SeS}{\textcolor{black}{7\spadesuit}}   
\newcommand{\ES}{\textcolor{black}{8\spadesuit}}   
\newcommand{\NS}{\textcolor{black}{9\spadesuit}}    
\newcommand{\TeS}{\textcolor{black}{10\spadesuit}}    
\newcommand{\JS}{\textcolor{black}{J\spadesuit}}    
\newcommand{\QS}{\textcolor{black}{Q\spadesuit}}   
\newcommand{\KS}{\textcolor{black}{K\spadesuit}}    
\newcommand{\AS}{\textcolor{black}{A\spadesuit}}     

\usepackage{xcolor}
% \usepackage{amssymb}
\usepackage{caption}
\usepackage{colortbl}
\usepackage{float} % for [H] placement, optional
\definecolor{myrowcolor}{HTML}{CCFFCC}  
% \usepackage[backend=bibtex]{biblatex}  % If using biblatex (more modern)
\newtheorem{theorem}{Theorem}
\usepackage{geometry}

% Configuring page margins to increase text width
\geometry{
  left=1.5in,
  right=1.5in,
  top=1in,
  bottom=1in
}


%%%%%%% END OF USEPACKAGE %%%%%%%%
\setlength{\parskip}{0.5em}
\newcommand{\py}[1]{\texttt{#1}}
\usetikzlibrary{decorations.markings}
\usetikzlibrary{decorations.pathmorphing}
\tikzset{
  midarrow/.style={
    decoration={markings, mark=at position 0.5 with {\arrow{>}}},
    postaction={decorate}
  }
}

\newcommand{\drawsmallboxwitharrows}[6]{%
    % Assume we are inside a tikzpicture already
    \begin{scope}[shift={(#5)}]
      % Draw the box with fill color
      \node[draw, fill=#6, minimum width=.1cm, minimum height=.1cm] (box) at (#1,0) {#4};
      
      % Draw the arrows
      \foreach \col/\x in {#2} {
        \draw[\col, thick, midarrow] 
          (#1+\x,0.7+\x/2) 
          to[out=270,in=90] 
          (box.north);
      }
    \end{scope}
}
\newcommand{\drawboxwitharrows}[6]{%
    % Assume we are inside a tikzpicture already
    \begin{scope}[shift={(#5)}]
      % Draw the box with fill color
      \node[draw, fill=#6, minimum width=.5cm, minimum height=.5cm] (box) at (#1,0) {#4};
      
      % Draw the arrows
      \foreach \col/\x in {#2} {
        \draw[\col, thick, midarrow] 
          (#1+\x,0.7+\x/2) 
          to[out=270,in=90] 
          (box.north);
      }
    \end{scope}
}



\tikzset{
  midarrow/.style={
    decoration={markings, mark=at position 0.5 with {\arrow{>}}},
    postaction={decorate}
  }
}

\usetikzlibrary{patterns} 

% % H (red)
\newcommand{\ThH}{\textcolor{red}{3\heartsuit}}  
\newcommand{\TwH}{\textcolor{red}{2\heartsuit}}  
\newcommand{\FoH}{\textcolor{red}{4\heartsuit}}   
\newcommand{\FiH}{\textcolor{red}{5\heartsuit}}   
\newcommand{\SiH}{\textcolor{red}{6\heartsuit}}    
\newcommand{\SeH}{\textcolor{red}{7\heartsuit}}  
\newcommand{\EH}{\textcolor{red}{8\heartsuit}}  
\newcommand{\NH}{\textcolor{red}{9\heartsuit}}   
\newcommand{\TeH}{\textcolor{red}{10\heartsuit}}   
\newcommand{\JH}{\textcolor{red}{J\heartsuit}}    
\newcommand{\QH}{\textcolor{red}{Q\heartsuit}}   
\newcommand{\KH}{\textcolor{red}{K\heartsuit}}    
\newcommand{\AH}{\textcolor{red}{A\heartsuit}}     

% D (red)
\newcommand{\ThD}{\textcolor{red}{3\diamondsuit}}  
\newcommand{\TwD}{\textcolor{red}{2\diamondsuit}}  
\newcommand{\FoD}{\textcolor{red}{4\diamondsuit}}   
\newcommand{\FiD}{\textcolor{red}{5\diamondsuit}}   
\newcommand{\SiD}{\textcolor{red}{6\diamondsuit}}    
\newcommand{\SeD}{\textcolor{red}{7\diamondsuit}}  
\newcommand{\ED}{\textcolor{red}{8\diamondsuit}}  
\newcommand{\ND}{\textcolor{red}{9\diamondsuit}}   
\newcommand{\TeD}{\textcolor{red}{10\diamondsuit}}   
\newcommand{\JD}{\textcolor{red}{J\diamondsuit}}    
\newcommand{\QD}{\textcolor{red}{Q\diamondsuit}}   
\newcommand{\KD}{\textcolor{red}{K\diamondsuit}}    
\newcommand{\AD}{\textcolor{red}{A\diamondsuit}}     

% C (black)
\newcommand{\ThC}{\textcolor{black}{3\clubsuit}}   
\newcommand{\TwC}{\textcolor{black}{2\clubsuit}}   
\newcommand{\FoC}{\textcolor{black}{4\clubsuit}}    
\newcommand{\FiC}{\textcolor{black}{5\clubsuit}}    
\newcommand{\SiC}{\textcolor{black}{6\clubsuit}}     
\newcommand{\SeC}{\textcolor{black}{7\clubsuit}}   
\newcommand{\EC}{\textcolor{black}{8\clubsuit}}   
\newcommand{\NC}{\textcolor{black}{9\clubsuit}}    
\newcommand{\TeC}{\textcolor{black}{10\clubsuit}}    
\newcommand{\JC}{\textcolor{black}{J\clubsuit}}    
\newcommand{\QC}{\textcolor{black}{Q\clubsuit}}   
\newcommand{\KC}{\textcolor{black}{K\clubsuit}}    
\newcommand{\AC}{\textcolor{black}{A\clubsuit}}     

% S (black)
\newcommand{\ThS}{\textcolor{black}{3\spadesuit}}   
\newcommand{\TwS}{\textcolor{black}{2\spadesuit}}   
\newcommand{\FoS}{\textcolor{black}{4\spadesuit}}    
\newcommand{\FiS}{\textcolor{black}{5\spadesuit}}    
\newcommand{\SiS}{\textcolor{black}{6\spadesuit}}     
\newcommand{\SeS}{\textcolor{black}{7\spadesuit}}   
\newcommand{\ES}{\textcolor{black}{8\spadesuit}}   
\newcommand{\NS}{\textcolor{black}{9\spadesuit}}    
\newcommand{\TeS}{\textcolor{black}{10\spadesuit}}    
\newcommand{\JS}{\textcolor{black}{J\spadesuit}}    
\newcommand{\QS}{\textcolor{black}{Q\spadesuit}}   
\newcommand{\KS}{\textcolor{black}{K\spadesuit}}    
\newcommand{\AS}{\textcolor{black}{A\spadesuit}}     

% \newcommand{\drawboxwitharrows}[6]{%
%     % Assume we are inside a tikzpicture already
%     \begin{scope}[shift={(#5)}]
%       % Draw the box
%       \node[draw, minimum width=.5cm, minimum height=.5cm] (box) at (#1,0) {#4};
      
%       % Draw the arrows
%       \foreach \col/\x in {#2} {
%         \draw[\col, thick, midarrow] 
%           (#1+\x,0.7+\x/2) 
%           to[out=270,in=90] 
%           (box.north);
%       }
%     \end{scope}
% }